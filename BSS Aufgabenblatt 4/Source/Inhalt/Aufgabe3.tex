\section{Aufgabe 3 (Entwicklung)}

\subsection{Teilaufgabe A)}
\subsubsection{Aus wie vielen Systemen besteht eine Systemlandschaft mindestens,
wenn nicht direkt in einem produktiven System entwickelt werden soll? }

Eine SAP-Systemlandschaft muss aus mindestens 2 Systemen bestehen, wenn
gewährleistet werden soll, dass die Entwicklung neuer Komponenten nicht auf
Produktivitätsdaten zugreifen dürfen.

Folgende Auflistung stellt eine Minimalkonfiguration dar:
\begin{itemize}
  \item Entwicklngssystem
  	\begin{itemize}
  	  \item Customizing-Mandant
  	  \item Testing-Mandant
  	  \item Schulungs-Mandant
  	  \item Qualitätssicherungs-Mandant
  	\end{itemize}
  \item Produktivsystem
  	\begin{itemize}
  	  \item Produktions-Mandant
  	\end{itemize}
\end{itemize}

\subsubsection{Welche Probleme tauchen bei dieser
Minimalkonfiguration möglicherweise bei der Qualitätssicherung auf?}

In dieser Konfiguration wird auf das separate Qualitätssicherungssystem QAS
verzichtet. Änderungen an unabhängigen Daten, die im Customizing
durchgeführt werden, können die Tests in der Qualitätssicherung
behindern. Außerdem kann die Vollständigkeit der Transporte, aus dem
Customizing, nicht vollständig sichergestellt werden. 

\subsection{Teilaufgabe B)}
\subsubsection{Welche Rolle spielt ein Entwicklungsauftrag im Zusammenhang mit der Organisation von
Entwicklungsarbeiten?}

Der Entwicklungsauftrag erfordert die Organisation von Arbeiten an einem oder
mehreren Entwicklungssystemen um das Produktivsystem unberührt zu lassen.
Hierbei übernimmt der Transport Organizer die Versionskontrolle. Bei Freigabe
des Auftrags muss jeder Entwickler eine Dokumentation anlegen, welche Ziele,
Status und Besonderheiten beinhaltet. Veränderungen an Objekten werden
automatisch protokolliert.

\subsubsection{Welche Rolle spielt eine Aufgabe?} 
Eine Aufgabe ist eine einzelne Entwicklung, Korrektur oder Reparatur, welche mit
den Änderungsaufträgen transportiert wird.

\subsubsection{Wie bewerten Sie die Trennung zwischen Aufgabe und Auftrag?} 
Die Aufgabe beschreibt ein zu veränderndes Objekt im Auftrag, so ist eine
durchsichtige, strukturierte Vorgehensweise möglich. Die Trennung ermöglicht
also eine effizientere Bearbeitung der jewiligen Auftragen im
Entwicklungsauftrag.

\clearpage