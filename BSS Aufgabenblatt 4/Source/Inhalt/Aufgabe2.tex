\section{Aufgabe 2 (Workflow)}

\subsection{Teilaufgabe A)}
\textbf{Was versteht man unter einem Workflow? Geben Sie eine kurze Definition (max. 2
Sätze)}

Ein Workflow beschreibt einen arbeitsteiligen, meist wiederkehrenden
Geschäftsprozess. Durch ihn werden die Aufgaben, Verarbeitungseinheiten sowie
deren Beziehungsgeflecht innerhalb des Prozesses (z.B. Arbeitsablauf und
Datenfluss) festgelegt.

\subsection{Teilaufgabe B)}
\textbf{Warum ist die Integration eines Workflow (Management) Systems in eine
betriebswirtschaftliche Standardsoftware sinnvoll (kurze Begründung)?}

Ein Workflowsystem spart Zeit, Geld und erhöht die Qualität. Es gewährt
einen Überblick über die Dinge, die im Unternehmen passieren. So lassen sich
Auswertungen anfertigen und Prozesse optimieren.

\subsection{Teilaufgabe C)}
\textbf{Warum ist es sinnvoll, die Aufbauorganisation eines Unternehmens separat zu
modellieren? Begründen Sie mit einem kurzen Beispiel.}

Die Aufbauorganisation legt fest, für welche Position eine Datenauswertung
nötig ist, wer Kennzahlen und Statistiken braucht. Statistiken können
außerordentlich, nicht wiederkehrend benötigt werden, im Gegensatz zu denen im
Workflow- System erfassten Prozessen. Wenn der Vorstand beispielsweise einen
Vortrag über den Ausschuss der Produktion im letzten Monat berichten muss, so
ist es ein Ausnahmeprozess diese erfassten Daten für diese Personengruppe
vorzubereiten. Die Leitung der Fertigung wäre ein natürlicher Kandidat. 

\clearpage