\section{Aufgabe 2 (Leistungsarten anlegen – Details)}


\subsection{Teilaufgabe A)}
\textbf{Was sind Leistungsarten und wozu werden sie benutzt?}

Leistungsarten sind Organisationseinheiten innerhalb des Controlling-Bereichs,
welche ausgeführte Leistung der Kostenstelle klassifizieren. Als Maßeinheiten
werden sie eingesetzt, um interne Aktivitäten zuzuordnen.

\subsection{Teilaufgabe B)}
\textbf{Welche Leistungsarten fallen in der Fallstudie an? In welchen
Einheiten werden sie gemessen?}
\begin{itemize}
  \item M017 - Monatestunden - MON017
  \item W017 – Wartungsstunden – WAR017
\end{itemize}

\subsection{Teilaufgabe C)}
\textbf{Warum wird keine Leistungsart für die Kantine angelegt?}

Für die Kantine wird keine Leistungsart angegeben, da die Kantine auf die
Kostenstellen Montage und Wartung umgelegt wird.

\subsection{Teilaufgabe D)}
\textbf{Welche internen Verrechnungssätze ergeben sich in der Fallstudie für
die Leistungsarten? Mit Hilfe welcher Transaktion werden diese Verrechnungssätze im SAP-System berechnet (Name
und Code der Transaktion angeben)?}

Durch den Transaktion Code KSBL (Kostenstellen: Planungsübersicht) kann die
Verrechnung der Kostenstellen angestoßen werden, dazu muss jedoch der Hacken aus Testlauf entfernt werden.

\clearpage 