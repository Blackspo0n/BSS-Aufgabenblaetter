\newcolumntype{L}[1]{>{\raggedright\let\newline\\\arraybackslash\hspace{0pt}}m{#1}}
\newcolumntype{C}[1]{>{\centering\let\newline\\\arraybackslash\hspace{0pt}}m{#1}}
\newcolumntype{R}[1]{>{\raggedleft\let\newline\\\arraybackslash\hspace{0pt}}m{#1}}



\setlength{\extrarowheight}{2pt}
\setlength{\tabcolsep}{0.3em}
\begin{tabular}{|L{3.5cm}L{9cm}L{3.5cm}|}
\rowcolor{heading}\color{white}\textbf{Aktivität} &
\color{white}\textbf{Erläuterung} &
\color{white}\textbf{SAPTransaktion(en)}\\
\hline
\textbf{Überprüfung der Planung} & Alle Kostenstellen wurden mit Primärkosten belastet. Weiterhin sehen Sie  die  Gesamtbelastung  und die interne Verteilungsmenge in der Wartung  und Montage. & KSBL (Kostenstellen: Planungsübersicht) \\ \hline
\textbf{Anlegen der Umlage} & Legt eine Umlage auf eine Kostenstelle an. & KSUB (Umlage) \\ \hline
\textbf{Überprüfung der Umlageergebnisse} & Überprüft die die Ergebnisse der Umlage. & KSBL (Kostenstellen: Planungsübersicht) \\ \hline
\textbf{Tarifermittlung der beiden Leistungsarten} & Ermittelt die Tarife der Leistungsarten. & KSPI (Tarifermittlung \\ \hline
\textbf{Prüfen der Auswirkungen Tarifermittlung} & Überprüft die Ergebnisse der Tarifermittlung. & KSBL (Kostenstellen: Planungsübersicht) \\ \hline
\end{tabular}
