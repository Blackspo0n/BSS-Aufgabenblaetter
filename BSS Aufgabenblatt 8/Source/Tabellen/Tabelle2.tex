\newcolumntype{L}[1]{>{\raggedright\let\newline\\\arraybackslash\hspace{0pt}}m{#1}}
\newcolumntype{C}[1]{>{\centering\let\newline\\\arraybackslash\hspace{0pt}}m{#1}}
\newcolumntype{R}[1]{>{\raggedleft\let\newline\\\arraybackslash\hspace{0pt}}m{#1}}



\setlength{\extrarowheight}{2pt}
\setlength{\tabcolsep}{0.3em}
\begin{tabular}{|L{3.5cm}L{9cm}L{3.5cm}|}
\rowcolor{heading}\color{white}\textbf{Aktivität} &
\color{white}\textbf{Erläuterung} &
\color{white}\textbf{SAPTransaktion(en)}\\
\hline

Planauftrag in Fertigungsauftrag umwandeln & Änderung von Planungsauftrag in Fertigungsauftrag. Direkte Anzeige des umgewandelten Planungsauftrags als Fertigungsauftrag. & MD04 (Bedarfs-/Bestandsliste) \\ \hline
Wareneingang ins Lager buchen & Buchung der Wareneingänge im Lager. Dabei werden neue Materialien erfasst und deren Lagerort, sowie Anzahl festgelegt. & MB1C (Sonstige) \\ \hline
Warenausgang zum Fertigungsauftrag buchen & Buchung des Warenausgangs auf den Fertigungsauftrag. Dabei werden die einzelnen Komponenten und ihre Menge aufgelistet. & MB1A (Warenausgang) \\ \hline
Fertigungsauftragsstatus anzeigen & Einsehen des Satus, des Fertigungsauftrags, mit Hilfe der Fertigungsauftragsnummer. Weiter kann man dort Dokumente wie Material-und Finanzbelege einsehen. & CO03 (Anzeigen) \\ \hline
Produktionsfertigstellung rückmelden & Rückmeldung zur Fertigstellung des Fertigungsauftrages. Erst möglich nach Fertigstellung des Fertigungsauftrags. & CO15 (Zum Auftrag) \\ \hline
Wareneingang zum Fertigungsauftrag & Nach Fertigung des Produkts wird dieses in das Fertigerzeugnislager gebucht. Zudem erhalten wir alle Details zum Fertigungsablauf des Produktes. & MB31 (Wareneingang) \\ \hline
Kosten Fertigungsauftrag anzeigen & Übersicht über alle Kosten die beim Fertigungsauftrag entstanden sind. & CO03 (Anzeigen) \\ \hline
Kosten Fertigungsauftrag abrechnen & Ermittlung des Gewinns, durch Aufteilung der Kosten zu den einzelnen Kostenträgern. & KO88 (Einzelbearbeitung) \\ \hline
\end{tabular}