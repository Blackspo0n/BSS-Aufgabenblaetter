\newcolumntype{L}[1]{>{\raggedright\let\newline\\\arraybackslash\hspace{0pt}}m{#1}}
\newcolumntype{C}[1]{>{\centering\let\newline\\\arraybackslash\hspace{0pt}}m{#1}}
\newcolumntype{R}[1]{>{\raggedleft\let\newline\\\arraybackslash\hspace{0pt}}m{#1}}



\setlength{\extrarowheight}{2pt}
\setlength{\tabcolsep}{0.3em}
\begin{tabular}{|L{3.5cm}L{6cm}L{3.5cm}L{3.5cm}|}
\rowcolor{heading}\color{white}\textbf{Aktivität} &
\color{white}\textbf{Erläuterung} &
\color{white}\textbf{Verantwortliche/Beteiligte} &	
\color{white}\textbf{SAPTransaktion(en)}\\
\hline
Anzeigen Bestellung & Die auf ein Angebot bezogene Bestellung wird angezeigt. & Einkauf & ME23N (Anzeigen) \\ \hline
Buchen Wareneingang zur Bestellung & Die bestellten Waren werden bei Bestandseingang in den Wareneingang gebucht. Der Bestand erhöht sich und ein Finanzbelegt wird erstellt. & Lager, Buchhaltung & MIGO\_GR (WE zur Bestellung (MIGO)) \\ \hline
Prüfen physischer Wareneingang & Ein Überblick über den Bestand wird organisationsebenenübergreifend gewährt. & Lager & MMBE (Bestandsübersicht) \\ \hline
Anlegen und Buchen der ersten Lieferantenrechnung & Eine Verbindlichkeit gegenüber dem Lieferanten wird erstellt. Eine Lieferantenrechnung wird erstellt. & Buchhaltung & MIRO (Logistik-Rechnungsprüfung) \\ \hline
Buchen Zahlungsausgang & Verbindlichkeiten werden durch Zahlung beglichen. & Buchhaltung & F-53 (Buchen) \\ \hline
Anzeigen Kreditorensaldo & Anzeigen einer Soll- und Haben-Buchung, sowie Möglichkeit zur Bestätigung dieser. & Buchhaltung & FK10N (Salden anzeigen) \\ \hline
\end{tabular}