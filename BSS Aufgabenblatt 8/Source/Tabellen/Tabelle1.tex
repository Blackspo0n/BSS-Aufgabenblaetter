\newcolumntype{L}[1]{>{\raggedright\let\newline\\\arraybackslash\hspace{0pt}}m{#1}}
\newcolumntype{C}[1]{>{\centering\let\newline\\\arraybackslash\hspace{0pt}}m{#1}}
\newcolumntype{R}[1]{>{\raggedleft\let\newline\\\arraybackslash\hspace{0pt}}m{#1}}



\setlength{\extrarowheight}{2pt}
\setlength{\tabcolsep}{0.3em}
\begin{tabular}{|L{3.5cm}L{9cm}L{3.5cm}|}
\rowcolor{heading}\color{white}\textbf{Aktivität} &
\color{white}\textbf{Erläuterung} &
\color{white}\textbf{SAPTransaktion(en)}\\
\hline
Materialstammsatz ändern & In der Sicht Disposition 3 müssen die Felder Strategiegruppe, Verrechnungsmodus und VerInt Rückwärts gepflegt werden & MM02 \\ \hline
Arbeitsplan ändern & Zuordnung der Komponenten zu den einzelnen Arbeitsschritten, dies kann für jeden Komponenten einzeln erfolgen, es ist aber auch mehrfach Auswahl möglich. Bei diesem Prozess ist jeder Prozessschritt von dem vorangegangenen abhängig. & CA02 (Ändern) \\ \hline
Produktgruppe anzeigen & Anzeigen der Unterteilung von Produkten in Produktgruppen. Dies hilft beim Planen auf höheren Aggregationsebenen. & MC85 (Anzeigen) \\ \hline
Absatz- und Produktionsgrobplan anlegen  & Planungswerkzeug, welches der Absatz- bzw. Produktionsplanung dient. Somit plant das System auf Grundlage der Vergangenheitswerte die Menge an Produkten, welche gefertigt werden muss. 

Werkzeug zur Planung der Produktion im Bezug auf erwarteten Absatz.
Die Planung kann dabei auch grafisch dargestellt werden. & MC82 (Ändern)
Version A00 \\ \hline
Absatz-/Grobplanung zu Programmplanung übergeben & Die Programmplanung ermöglicht es eine detaillierte, statt lediglich einer groben Planung vor zu nehmen. Dabei wird die entsprechende Produktgruppe auf die individuellen Komponenten herunter gebrochen, welche zur ausgewählten Gruppe gehören. Der Planer ist hierbei in der Lage, die Ergebnisse der Planung zu verändern, bevor er diese (manuell) sichert. & MC75 (Überg. Progr. Pl. PG) \\ \hline
Programmplanung anzeigen & Überprüfung des Planprimärbedarfs der Produktgruppen.  & MD63 (Anzeigen) \\ \hline
Leitteileplanung und Materialbedarfsplanung starten & Planung der Leitteile und des Materialbedarfs. Nach Planung werden alle Parameter zu der Planung angezeigt. & MD41 (Einzelpl. Mehrstufig) \\ \hline
Bedarfs/Bestandsliste anzeigen & Übersicht alle Planungs-und Fertigungsaufträge zum gewählten Material. Sowie graphische Anzeige der Einzelaufträge.
 & MD04 (Bedarfs-/Bestandsliste) \\ \hline
 \end{tabular}


