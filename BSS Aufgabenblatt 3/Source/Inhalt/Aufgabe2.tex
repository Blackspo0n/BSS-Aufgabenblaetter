\section{Aufgabe 2 (Berechtigungskonzept)}
\label{sec:Aufgabe 2 (Berechtigungskonzept)}

\subsection{Teilaufgabe A)}
\textbf{Das Berechtigungskonzept einer betriebwirtschaftlichen
Standardsoftware muss zwei sich widersprechende Anforderungen erfüllen. 
Welche beiden Anforderungen sind dies und warum widersprechen sich die
Anforderungen?}\\

Im SAP-System kann man zwischen zwei Anforderungen unterscheiden:

\begin{itemize}
  \item \textbf{Berechtigungen müssen sehr fein steuerbar sein}, da die
  Zuordnung von Funktionen zu Stellen oder Personen, in unterschiedlichen
  Unternehmen, sehr voneinander abweichen kann. Dies betrifft einerseits die
  generelle Durchführbarkeit bestimmter Funktionalitäten durch bestimmte Stellen im
  Unternehmen, andererseits aber auch die Durchführung einer Funktion nur für
  bestimmte Teilbereiche eines Unternehmens.
  
  \item \textbf{Die Berechtigungsvergabe muss einfach sein}, wenn sichergestellt
  werden soll, dass Berechtigungen nur bedarfsgerecht vergeben werden. Konkret
  bedeutet dies, dass möglichst nur sehr wenige Berechtigungen, mit klar
  erkennbarer Bedeutung, an einen Mitarbeiter vergeben werden. Alternativ kann
  die Berechtigungsvergabe automatisch anhand eines Stellenprofils vergeben
  werden. Dies setzt aber voraus, dass ein entsprechendes Profil in maschineller
  Form vorliegt.
\end{itemize}

Die beiden Anforderungen stehen im Konflikt. 
Denn je feiner die Rechteverwaltung einstellbar ist,
desto komplexer wird diese auch. Diesen Konflikt kann man zwar mit Mitteln wie
\textbf{profiling} wieder entgegenwirken, trotzdem bleibt immer eine gewisse
Komplexität erhalten.\\

Es gilt also einen Kompromiss zwischen eben diesen beiden Anforderungen zu
finden.
\clearpage

\subsection{Teilaufgabe B)}
\textbf{Über welchen Berechtigunsprofile verfügt der von Ihnen verwendete
SAP-Benutzer?(Anzeigen z.B. Werkzeuge -> Administration -> Benutzerpflege ->
Benutzer) Fügen Sie einen entsprechenden Screenshot ein.}\\[2ex]

\textbf{Schauen Sie sich das erste dort hinterlegte Profil an (Doppelklick auf
Profilname). Klappen Sie die erste Ebene auf und fügen Sie einen entsprechenden
Screenshot hinzu.
}

Screenshot aus der Benutzerverwaltung:
\begin{center}
\includegraphicsKeepAspectRatio{profilpage.png}{1}
\end{center}
\clearpage

Screenshot aus der Benutzerverwaltung über die Rechte des eigenen Nutzers:
\begin{center}
\includegraphicsKeepAspectRatio{profilpage_bernhardt.png}{1}
\end{center}