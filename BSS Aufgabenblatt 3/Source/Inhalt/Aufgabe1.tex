\section{Aufgabe 1 (Klassifikation)}
\label{sec:Aufgabe 1 (Klassifikation)}
\textbf{Gegeben ist eine Tabelle\footnote{Zu entnehmen aus dem Aufgabenblatt},
die drei verschiedene Automodelle beschreibt. \\[1ex]
Welche Klasse(n), Merkmale und Merkmalsausprägungen bilden Sie bei Benutzung
der Klassifikation? Woran liegen die Besonderheiten bei der Beschreibung der
wählbaren Extras?}

Wenn die Klassifizierung angewendet wird enthalten wir die Klasse
\textbf{"PKW"}. Folgende Merkmale können dabei herausgearbeitet werden:

\tabelle{Merkmale}{tab:markmale}{Merkmale.tex}

Mögliche Ausprägungen der Merkmale sind:
\tabelle{Ausprägungen}{tab:auspraegung}{Auspraegungen.tex}

Das Merkmal \textbf{Extras} kann mehrere Merkmalsausprägungen annehmen. Um
dieses Verhalten im SAP-System korrekt abzubilden, muss das Merkmal
mit der Option \textbf{Mehrwertig} angelegt werden.\\

Um die Klassifizierung abzuschließen, müssen die Merkmalsausprägungen den
Merkmalen und die Merkmale der Klasse hinzugefügt werden.
