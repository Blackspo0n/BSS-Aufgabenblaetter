\newcolumntype{L}[1]{>{\raggedright\let\newline\\\arraybackslash\hspace{0pt}}m{#1}}
\newcolumntype{C}[1]{>{\centering\let\newline\\\arraybackslash\hspace{0pt}}m{#1}}
\newcolumntype{R}[1]{>{\raggedleft\let\newline\\\arraybackslash\hspace{0pt}}m{#1}}



\setlength{\extrarowheight}{2pt}
\setlength{\tabcolsep}{0.3em}
\begin{tabular}{|L{3.5cm}L{6cm}L{3.5cm}L{3.5cm}|}
\rowcolor{heading}\color{white}\textbf{Aktivität} &
\color{white}\textbf{Erläuterung} &
\color{white}\textbf{Verantwortliche/Beteiligte} &	
\color{white}\textbf{SAPTransaktion(en)}\\
\hline
Anlegen neuer Kunde &
Die Stammdaten für einen neuen Kunden werden angelegt.
Dabei werden Daten für Vertrieb und Buchhaltung erfasst. &
Vertrieb, Buchhaltung &
XD01(Gesamt)
XD01(Nur Vertrieb)\\
\hline
Anlegen Ansprechpartner für Kunde &
Es wird ein Ansprechpartner festgelegt, welcher Mitarbeiter des
Kundenunternehmens ist. Die Daten des Ansprechpartners werden erfasst (darunter
auch seine Position im Unternehmen und seine Abteilung). & Vertrieb & VAP1
(Anlegen)\\
\hline
Ändern Kunde &
Nachträgliche Änderung der Kundenstammdaten.&
Vertrieb & VD02 (Nur Vertrieb)
XD02(Gesamt)\\
\hline
Anlegen Kundenanfrage &
Erstellen einer Anfrage, diese ist dabei unverbindlich und kann Material oder
Dienstleistungen umfassen. & Vertrieb & VA11 (Anlegen) \\
\hline
Anlegen Angebot &
Erstellen eines verbindlichen Angebots, ähnlich des Anlegens einer
Kundenanfrage. & Vertrieb & VA21 (Anlegen) \\
\hline
Anlegen Kundenauftrag mit Bezug auf Angebot &
Das bereits existierende Angebot wird in einen Auftrag umgewandelt.& Vertrieb &
VA01 (Anlegen) \\
\hline
Anzeigen Kundenauftrag & Anzeigen des erstellten Kundenauftrags um diesen im
Detail zu überprüfen. & Vertrieb & VA03 (Anzeigen) \\
\hline
Anlegen Auslieferung &
Erstellung eines Auslieferungsbelegs um den Lieferprozess beginnen zu können. &
Vertrieb, Versand und Transport & VL01N (mit Bezug auf Kundenauftrag) \\
\hline
Kommissionieren Material & Änderung der Daten des Auslieferungsbelegs. &
Vertrieb, Versand und Transport & VL02N (Einzelbeleg) \\
\hline
Anzeigen Belegfluss & Zugriff auf das Belegfluss-Werkzeug, mit welchem alle
Dokumente zum Auftrag verbunden sind. & Vertrieb, Versand und Transport, Buchhaltung (Alle an dem Auftrag beteiligte Bereiche) & VA03 (Anzeigen) \\
\hline
\end{tabular}
