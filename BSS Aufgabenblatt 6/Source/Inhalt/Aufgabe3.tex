\section{Aufgabe 3 (Auftrag)}
\subsection{Welche konkreten Informationen werden für jeden Auftrag ermittelt?}

Die folgenden Informationen werden für jeden Auftrag ermittelt:
\begin{itemize}
  \item Konkrete Termine
  \item Versandstelle und Routenfindung
  \item Verfügbarkeitsprüfung
  \item Transfer von Anforderungen an MRP
  \item Preisfindung
\end{itemize}


\subsection{Bei der Lieferterminermittlung kann eine Vorwärts- oder eine
Rückwärtsterminierung verwendet werden. Beschreiben Sie beide Vorgehensweisen kurz (gerne auch mit Skizze).}

\subsubsection{Rückwärtsterminierung}
Bei dieser Terminierungsart terminiert das System rückwärts ausgehend vom
Eckendtermin/terminierten Ende des Auftrags.
\begin{center}
\includegraphicsKeepAspectRatio{Bilder/Auswahl_027.png}{0.7}
\end{center}

\subsubsection{Vorwärtsterminierung}
Bei dieser Terminierungsart terminiert das System vorwärts ausgehend vom Eckstarttermin/terminierten Start des Auftrags.
\begin{center}
\includegraphicsKeepAspectRatio{Bilder/Auswahl_028.png}{0.7}
\end{center}
\clearpage