\addsec{Aufgabe 1}
\label{sec:Aufgabe 1}

Die Anpassung einer Standardsoftware auf die eigenen Bedürfnisse erfolgt durch die drei Möglichkeiten, Anpassbarkeit, Erweiterbarkeit und Modifikation.\\
\\
Anpassbarkeit einer Standardsoftware meint, dass diese in unterschiedlichen
Situationen eingesetzt werden kann. Diese Anpassbarkeit liegt aber nur vor, wenn
die entsprechende Einsatzsituation vom Hersteller angedacht ist. Somit sind die
nötigen Ressourcen zur Anpassung der Software bereits enthalten. Alle getroffenen Anpassungen sind dabei versions- und releasesicher. Beim SAP ERP meint Anpassbarkeit die Maßnahmen, welche durch Customizing oder Personalisierung erreicht werden können. Anpassbarkeit ist vereinbar mit dem Nutzen von Standardsoftware, denn es werden lediglich die Möglichkeiten, welche durch die Software geboten werden, ausgeschöpft und der Support bleibt dabei erhalten. \\
\\
Hinzu kommt noch die Erweiterbarkeit, bei welcher die gewünschte Funktionalität,
im Gegensatz zur Anpassbarkeit, in der Software nicht vorhanden ist. Zur
Erweiterung muss eine Erweiterbarkeit der Software vorliegen, denn dieser werden
eigene Ressourcen hinzugefügt. Beim SAP ERP wird die Erweiterbarkeit durch eine integrierte Entwicklungsumgebung unterstützt. Ähnlich wie die Anpassbarkeit, ist auch die Erweiterbarkeit vereinbar mit dem Nutzen von Standardsoftware, denn diese greift auf eine speziell dafür geschaffene Schnittstelle zu und erweitert dabei sogar den Nutzen der Software. Zudem bleiben der Support und sämtliche Funktionen der Software enthalten. \\
\\
Wenn eine Funktion grundsätzlich in der Standardsoftware enthalten ist, aber
dies nicht auf die eigenen Bedürfnisse, mit Hilfe der Anpassbarkeit, angepasst
werden kann, dann wird die Modifikation verwendet. Eine Modifikation wird durch SAP ERP unterstützt, ist aber meist, im Gegensatz zur Anpassbarkeit, nicht versions- und releasesicher. Jedoch wird von SAP verlangt, dass der Benutzer alle Modifikationen registriert, wodurch der Support für modifizierte Bestandteile aus Kundensicht problematisch ist. Im Gegensatz zur Anpassbarkeit und Erweiterbarkeit, ist die Modifikation nur teilweise mit dem Nutzen einer Standardsoftware vereinbar, denn das bereits existierende Konzept der Software wird nicht erweitert, sondern durch einen Eingriff in die vorgefertigte Software modifiziert. Diese Modifikation erfolgt nicht im Rahmen der eigentlich gedachten Anpassung der Software. Zudem geht dabei der Support an dieser Stelle verloren, was bei möglichen Komplikationen schnell zu Problemen führen kann.
