\addsec{Aufgabe 2}
\label{sec:Aufgabe 2}
 
Beim Customizing wird die Standardsoftware, vom Auslieferungszustand aus, den
Bedürfnissen des Anwenders angepasst, dies erfolgt ohne Änderung des Quellcodes.
Diese Anpassung erfolgt in drei, fest vom Hersteller der Software, festgelegten
Schritten (bei Fragen kann die Hilfe-Funktion in Anspruch genommen werden). Der
erste Schritt umfasst die Anpassung an die landesspezifischen Einstellungen, wie
Sprache und Währung. Im zweiten Schritt sollen die betrieblichen Organisations-,
Funktions- und Datenstrukturen des Unternehmens abgebildet werden. Der dritte und letzte Schritt befasst sich mit der Abbildung der betrieblichen Prozesse.\\[4ex]
\textbf{Customizing der Unternehmensstruktur im Finanzwesen:}
\begin{itemize}
	\item Unternehmensstruktur – Allgemeine Einstellung
	\begin{itemize}
	  \item Auswahl der Struktur
	  \item Einstellung der Landeseinstellungen und Landesspezifikationen
	\end{itemize}
	\item Definition
	
	\begin{itemize}
	  \item Gesellschaft definieren
	  \item Kreditkontrollbereich definieren
	  \item Buchungskreis bearbeiten, kopieren, löschen, prüfen
	  \item Geschäftsbereich definieren
	\end{itemize}
	\item Abbildung der betrieblichen Prozesse und einpflegen der Daten
\end{itemize}
